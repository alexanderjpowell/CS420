%%%%%%%%%%%%%
% 
% Alexander Powell
% Human Computer Interface and Design
% Team Project Assignment #1
% 02.29.2016
% 
%%%%%%%%%%%%%

\documentclass[11pt]{article}

\usepackage{times,mathptm}
\usepackage{pifont}
\usepackage{exscale}
\usepackage{latexsym}
\usepackage{amsmath}
\usepackage{amssymb}
\usepackage{amsthm}
\usepackage{epsfig}
\usepackage{tikz}
\usepackage{textcomp}
\usepackage{enumerate}


\textwidth 6.5in
\textheight 9in
\oddsidemargin -0.0in
\topmargin -0.0in

\parindent 15pt     % How much the first word of a paragraph is indented. 
\parskip 1pt	   % How much extra space to leave between paragraphs.

\begin{document}

\begin{center}             % If you're only centering 1 line use \centerline{}
\begin{LARGE}
{\bf Team Project Assignment \# 1}
\end{LARGE}
\vskip 0.25cm      % vertical skip (0.25 cm)

Due: Monday, February 29 \\  % force new line
Alexander Powell
\end{center}




\item For most student projects, it will be difficult to conduct real stakeholder interviews, so let’s 
assume you did interviews with the people you listed under item 3. and now you need to 
write up your findings at the end of the day in a brief report. Please write brief report based 
on your assumptions of what would come out of stakeholder interviews. 

There are two major stakeholders for the EatWithFriends application.  The first are the students who are the main users of the app.  The second group is William & Mary dining/Sodexo (or whoever is the current dining provider).  In this case, the students are the primary stakeholder and the administration is a secondary stakeholder.  

For the interview with W&M students, we asked them what they would like to see in the application.  One of the biggest concerns was that the menu information, including details lists of recipe ingredients were easy to find.  This is a big concern for those students with allergies to certain foods.  Additionally, students wanted to have different settings for seeking friends that were already eating in their area, as well as the option to publicly broadcast your location and meet new people.  Others voiced concern about the method of possibly finding strangers to eat with.  They made the point that there are probably individuals that you don't want to eat with.  From this feedback, it seems like a good idea to be able to "block" people from seeing your profile if you are looking for friends to eat with.  

Next, we had our interview with members of both the W&M Dining administration and Sodexo.  In this case, this stakeholder is not directly using the app but instead taking data from it.  They seemed to like the idea of having more dialogue for student reviews of the food and service.  Since a lot of the value of EatWithFriends comes from giving up-to-date info on dining menus, we were concerned about how easy it would be to always display that on the app.  Sodexo officials told us that the menus were updated daily, before each meal.  They said that any changes should be placed directly on the site, so if our app scrapes this info from the site we should always be showing the current info.  Another aspect that was important to the dining providers was to provide them with more data about the students they server.  They asked if we could use the app to collect data on when most students eat, where they go, and what they like and dislike.  They also mentioned the possibility of having a recommendation section or "suggestion box", inside the app somewhere to help them provide the best service.  





\end{document}




































