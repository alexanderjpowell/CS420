%%%%%%%%%%%%%
% 
% Alexander Powell
% Human Computer Interface and Design
% Homework Assignment #1
% 01.31.2016
% 
%%%%%%%%%%%%%

\documentclass[11pt]{article}

\usepackage{times,mathptm}
\usepackage{pifont}
\usepackage{exscale}
\usepackage{latexsym}
\usepackage{amsmath}
\usepackage{amssymb}
\usepackage{amsthm}
\usepackage{epsfig}
\usepackage{tikz}
\usepackage{enumerate}


\textwidth 6.5in
\textheight 9in
\oddsidemargin -0.0in
\topmargin -0.0in

\parindent 15pt     % How much the first word of a paragraph is indented. 
\parskip 1pt	   % How much extra space to leave between paragraphs.

\begin{document}

\begin{center}             % If you're only centering 1 line use \centerline{}
\begin{LARGE}
{\bf CSci 420 Homework \#1}
\end{LARGE}
\vskip 0.25cm      % vertical skip (0.25 cm)

Due: Sunday, Jan 31\\  % force new line
Alexander Powell
\end{center}

\begin{enumerate}
\item
Task 1
\begin{enumerate}

\item
\textbf{Problem Statement:}
This project will provide a more organized source of information on dining hall food on campus.  Currently, there doesn't seem to be a very good system to look-up the dining hall menus or read student reviews of the food.  This project would attempt to provide those resources to students in one easy to use application.  

\item
\textbf{Value Proposition:}
This project will be a better source of information for students to decide where and what they want to eat on campus.  

\item
\textbf{Feedback:}
Some students I spoke to voiced concerns about low student participation.  Others seemed excited about a revised menu system and the ability to leave comments and criticisms about the food.  

\item
\textbf{Design Brief:}

\setlength{\parindent}{10ex} Most people would consider food one of the most important aspects of their lives, especially college students.  At a school like William \& Mary, with several different dining halls on campus to choose from and a limited number of meal swipes, it's important that you make the right choice for each meal.  It's disappointing when you go to have lunch at Sadler only to learn that they were serving mac and cheese at the Caf.  Also, it would be very useful if there was some easy to access source of student food reviews to have the best dining hall experience.  While the school does post daily menus online, from my experience no one seems to bother checking them.  If there was a more convenient way to find this information, on a simple mobile application for example, students would probably be much more involved.  Also, while other social networking sources like Yik Yak and Facebook exist, there is no outlet solely dedicated to a discussion on dining hall food.  

\indent The success of this project relies on student interest in the food that they eat, as students are the targeted audience for this app.  If the population is indifferent about what they're eating then the product will not be very popular and fail.  Additionally, some element of participation is required because the component of student reviews will only be useful if other students join the discussion.  The environment of this project is essentially limited to college campuses.  The actual functionality of the project is quite feasible and simple to construct.  The biggest risk of failure would be simply not gaining enough popularity to make it worth using.  To be effective, this project requires input from all students.  Another concern is how accurate the dining provider is at updating menus and correctly listing ingredients.  The major processes of the dining project would be to pull menus and other info from the school's website and display it in an easy to read and informative way, possibly displaying ingredient lists and different icons to signify ingredients that people are commonly allergic to.  The other major component provides a system of customer reviews where students can leave comments regarding the quality of the food they tried.  This will most likely be anonymous and there will be options to flag inappropriate responses.  Also, these reviews will auto-delete after a 3 hour period so that the feed will stay up to date.  

\end{enumerate}

\newpage

\item
Task 2

\begin{enumerate}[1]
\item
\textbf{Problem Statement:}
The problem with the current course registration system is that it fails to provide a simply and easy way to quickly lookup and add classes to your schedule.  With the current system, classes fill up too quickly and you have to rely on a lot of backup courses.  Also, the design itself can be rather difficult to navigate.  

\item
\textbf{Value Proposition:}
Taken from slides: \textit{"MyCourseRegistrator is an app for students to register for their classes that is convenient, stressless and provides all students a fair chance to get into high demand courses."}

\item
\textbf{Feedback:}
Many other students seemed excited about a new course registration system, citing flaws in the current system about how classes fill up too quickly.  Some brought up possible difficulties in actually implementing the new system, as it would require participation from the school.  

\item
\textbf{Design Brief:}

\setlength{\parindent}{10ex} The philosophy behind this project is the belief that it's possibly to create a better course registration.  The current process is outdated and difficult to work with.  Our goal is to redesign the current process so that students will be less frustrated by the whole process.  For the structure of the project, it's important to note that the school will have to be involved in the development of this product in some way, as they are the ones who keep track of student records and transcripts.  Most likely it will be the result of collaboration between students and school officials.  The new course registration will provide a more streamlined method to register for the courses you need to graduate by providing one main page with all the information about classes listed in a big table.  This will include basic info like the professor's name, class times, number of spots remaining as well as more advanced options like a button to request to be notified via text or email when a class becomes open.  Also, it would be easier if you could add a class from the browse page instead of copying down a CRN number and typing it into a different window.  As the whole process will remain online, it's environmental effect will be minimal.  

\indent There are some serious concerns about how feasible a redesign would be.  For one, the school would need to be involved to change the official way that students register for classes.  Also, most of the changes from the old system would have to do with the visual design of the interface as opposed to actually changing the method of registration.  A better answer to students complaining that they can't get the classes they want would be to open more sections and hire more professors.  When it comes down to functionality, the current method is lacking in several areas.  It would be helpful if instead of writing down all the CRN numbers and then navigating to a different page, you were able to register for a class with just one click.  This would definitely be simpler than having multiple tabs in several different windows open.  Additionally, it would be nice if the new registration system could link the teacher's score on RateMyProfessor.com.  Furthermore, another feature that can be added would be to send out emails to students who have requested to be notified if more spots in a class opened up.  


\end{enumerate}

\end{enumerate}

\end{document}




































